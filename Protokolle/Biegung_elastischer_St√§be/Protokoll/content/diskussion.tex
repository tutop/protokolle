\section{Diskussion}
\label{sec:Diskussion}
Mögliche Ursachen für die Fehler sind die Stäbe selbst. Durch ihren häufigen Gebrauch ist es möglich das eine natürliche Biegung vorliegt, die sich vermutlich in den
Messergebinssen für die zweiseitige Einspannung niederschlägt. Die großen Abweichung der Messpunkte, die weiter außen gemessen wurden, spricht für eine solche
Verbiegung des Stabes. Dies kann so behauptet werden, da, wie später diskutiert wird, die bestimmten Elastizitätsmodule des Versuchs der zweiseitigen Auflage gut übereinstimmen
mit dem Versuch bei einseitiger Einspannung. Eine weitere Fehlerquelle in der Aufhängung des Gewichts liegen, welche nicht fest am Stab ist sondern lose und dadurch leicht
verutscht beim Ein- und Aushaken der Gewichts.
\\
\\
Hier noch einmal die ermittelten Elastizitätsmodule im Überblick:
\begin{align*}
  E_1 &= \SI{1.2745+-0.0006e11}{\frac{\newton}{\meter^2}} \\
  E_2 &= \SI{1.99+-0.07e11}{\frac{\newton}{\meter^2}} \\
  E_3 &= \SI{2.1218+-0.0011e11}{\frac{\newton}{\meter^2}} \\
  E_4 &= \SI{1.8863+-0.0009e11}{\frac{\newton}{\meter^2}} \\
\end{align*}
 

Aus der Bestimmung der Dichte der benutzten Stäbe ergibt sich für den Stab mit rechteckigem Querschitt
\begin{align*}
  \rho_1  &= \SI{8377.6+-4.2}{\frac{\kilo\gram}{\meter^3}}
  \intertext{und für den zylindrischen Stab}
  \rho_2  &= \SI{7920+-0.4}{\frac{\kilo\gram}{\meter^3}}
\end{align*}

Diese Werte ergeben sich aus den Abmessungen der beiden Stäbe.
Bei dem Stab mit dem rechteckigem Querschnitt wird es sich vermutlich um ein stahl- oder eisenhaltiges Material handeln. Der zylindrische Stab wird hingegen wahrscheinlich
ausschließlich aus Eisen sein.
